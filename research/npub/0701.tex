\documentclass{letter}
\begin{document}

Sir,

This is the formal appeal against the rejection of my article
 LM9256, which the editor finds not suitable to process.

I am afraid that my statement {\it "it is not the goal of this note
to reveal the answer, just we aim to state the problem} induces the
reader to think that I am disregarding my duties as author. It
is not the case. Although it is true that nuclear matter is beyond
my usual field of research, so I can not honestly aim to give a closed
answer to the problem. In this sense,
in a letter attached to the copyright transfer form, I included
a self-evaluation of my letter which could be relevant.

\begin{quote}
Before starting the process of referring, I'd like to forward you
my own evaluation on the subjective points you ask referees -and
indirectly, authors- to consider: Validity, Importance and Interest. For
my own paper, I would score high to the first and third points, and low
or very low to the second one.

\begin{itemize}

\item - Is the work scientifically sound?  If not, do you believe the paper
 can be revised to correct the scientific defects you find?

Yes, it is sound. It reports a fact, and it examines the models that
can be related to it.

\item -  Does the manuscript report substantial research?  Is the conclusion
   very important to the field to which it pertains?

It could be depending of further work, as if confirmed it indicates the need
of raising the cut-offs used in nuclear calculations.

\item  Is the research
   at the forefront of a rapidly changing field?  Will the work have a
   significant impact on future research?

No. Nuclear models have an independent pace, and Higgs research depends
mostly in the CERN LHC for small higgs masses (the Tevatron can prove
unsuitable for mass<135, as a recent PRL shows). The effect reported in this
paper will not forseeably modify the trends in ab-initio calculations
for nuclear force, but if it could be introduced in a remote future, specially
if the Higgs mass is confirmed.

\item {- Is this paper of broad interest?}

Yes as if true, it will become a common textbook remark. And with enough 
computing power, it can provide independent confirmation of SM masses. 
Besides, the questions raised in the paper could be answered by physicists 
working in other branches of quantum physics, not necessarily nuclear 
physicists.

\end{itemize}
So a YES/NO/YES should be my own answer to the subjective analysis. Of course 
the No could be turned into a Yes with further reshaping of the work during 
the refereeing process and an adequate propaganda campaign... In fact one of 
my goals in asking publication in PRL is to reach a wider public, so that the 
paper will not be disregarded at first stage, on grounds of excessive 
speculation.

In this sense, let me remark that your note to referees stress that PRL 
"welcome speculative ideas provided that their consequences and ramifications 
have been sufficiently well considered and, to the extent possible, have been 
spelled out." I have intended to do this, within the limits of reasonable 
hope, in the last section of the letter.  It is fortunate that the work is in 
some sense falsifiable at mid term, as if either no particle appears near 115 
GeV or QHD proves able to reproduce the spin couplings for double magicity, 
the coincidence will be invalid in the first case, accidental in the later. 
Thus a ten-years lapse will do the final judgement, as computing power 
increases and the LHC becomes to work, we will see how the letter evolves; it 
could be a PRL classic or to go down to simply three sheets of wasted paper 
space, but just the existence of
this lapse makes me believe that the appropriate timing, to publish it, is 
just now.

\end{quote}

The evaluation of the editor is very different. He states that
\begin{quote}
Unfortunately, in order for the editors to consider
your manuscript for review, it must at least demonstrate that the
underlying evidence has a high degree of certainty and that a robust
attempt has been made at exploring possible solutions in terms of
fundamental physics.
\end{quote}

It puzzled my because I think the evidence is solid (there is no more massive
bosons, no more doubly-magic numbers) and I in fact think that the solution
should be found in terms of known fundamental physics.

Perhaps the editor feels a touch of numerology and esoterism in the paper.
But usual esoterism is done with incongruent units, while numerology works
mostly with adimensional units. Here the evidence is shown in terms of
congruent {\it mass} units. It does not matter if you take uma or GeV or
kilograms, the plot is always the same.

The rejection letter follows:

\begin{quote}
And in this regard it would have been helpful if
your manuscript had presented a convincing case why a correlation of
one property between two quite different physical systems should
require a fundamentally new physics perspective.
\end{quote}

As I say, I do not claim a fundamentally new physics perspective.
The editor seems lured by the weak bond approximation. All HEP
experiments rely on this approximation, ie that the momentum scale
of the colliding particle is a lot bigger that the coupling between
the target particle and the rest of the system. This is not the case;
all my claim, if any, is that the weak bond approximation is not
valid in the calculation of ground nuclear states.

An analogous problem happens in classical mechanics: take two particles
joined by Hooke's law with coupling K. Suppose that a third particle impacts
against one of them. If K goes to zero, the momentum transfer will
depend essentially of the mass of the target particle. If K goes to
infinity, both particles are "pasted solid" and the momentum transfer
will depend of the mass of the whole system.

In our system the strong nuclear force does the role of Hooke's law,
against the weak interaction from our massive virtual bosons.

A second question could be if the interaction coupling of the
W or the Higgs can compete with the ones from pions. Here I will
suggest to check the conditions of perturbation of a yukawa potential
with another one, if the former bonds strongly the system. Then
by approaching $exp(x)\approx (1+x+...)$ it appears a term
containing the product of coupling constant times the boson mass.

If you like, I could incorporate this kind of arguments in the letter before
sending it to referees; I neglected it because I find they are easy enough for
anyone working out the problem and I was afraid that them could drive to
confusion if taken literally instead that simply as an analogy.

Yours,

Alejandro Rivero

\end{document}
